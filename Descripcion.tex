%%%% Descrpción del proyecto

\chapter{Descripción del Proyecto}

%%% AREGLAR 
%Justificacion del proyecto
\section{Justificación del proyecto}

Estamos en la era donde la comunicación digital por medio de internet está incrementando; por esta razón surge con más fuerza el Internet de las Cosas (Internet of  Things o IOT), con sus múltiples aplicaciones en ciencia y tecnología \cite{Noguera2016}. El  IoT es un concepto con el cual el mundo real interactúa con la tecnología que cada día se desarrolla para facilitar la vida cotidiana \cite{Atzori20102787};  por esta razón cada vez más personas estan conectadas a la \textit{WEB}, lo que causa que el incremento en la conectividad sea necesaria, además del volumen de almacenamiento de los servidores para soportar la capacidad de usurarios ingresando y que no afecte el funcionamiento de las plataformas \textit{online} \cite{10055838620160101}; para esto cada vez se desarrollan nuevas tecnologías donde la velocidad de trasferencia de datos y la capacidad son primordiales. 
\\
%Con esto llegamos a decir que la integración de diferentes tecnologías es la forma en como la sociedad en este momento está buscando solución a los problemas tanto de comunicación como de aspectos tecnológicos y científicos \cite{Atzori20102787}. 
\\
Ya que IoT se puede emplear en muchos campos tecnológicos se plantea que el monitoreo de procesos por medio de este, mejoraría el control o manejo de los mismos, ya que se podría analizar en tiempo real, mejorando la respuesta ante posibles fallas. Por este motivo, ante la necesidad de los estudiantes de ingeniería electrónica que previamente diseñaron el prototipo de caracterización de paneles fotovoltaicos y de medición de variables meteorológicas\cite{Paneles}\cite{Proto}, de poder adquirir y visualizar las diferentes variables obtenidas en tiempo real mediante una plataforma \textit{online} y una aplicación móvil, facilitando de esta forma el uso de este prototipo, ya que con estas, se podrá manejar remotamente sin necesidad de su manipulación manual.

%Objetivos del proyecto
\section{Objetivos del Proyecto}

\subsection{Objetivo general}

Monitorear, transmitir y almacenar los datos obtenidos en tiempo real, mediante IoT, de un prototipo de caracterización de paneles fotovoltaicos y de medición de variables metereológicas.

\subsection{Objetivos específicos}

\begin{itemize}
\item Monitorear y trasmitir las variables medidas, en tiempo real por medio de un sistema de desarrollo digital. 
\item  Desarrollar una aplicación para móviles y una plataforma \textit{online}, que permita visualizar y almacenar las variables medidas en una bases de datos (nube o en un computador) con ayuda de herramientas disponibles para el diseño de estas plataformas.
\item Realizar el montaje de un sistema de desarrollo con WIFI para comunicación del prototipo de caracterización de paneles fotovoltaicos y de medición de variables metereológicas.

\end{itemize}








