%%%% Descrpción del proyecto

\chapter{Descripción del Proyecto}

%%% AREGLAR 
%Justificacion del proyecto
\section{Justificación del proyecto}

El trabajo de investigación es una tarea que permite el desarrollo de nuevas tecnologías y es indispensable para el avance de la civilización en futuros cercanos y lejanos \cite{Noguera2016}  %hay que referenciar el libro de la física del futuro
Debido a esto es una tarea importante para un país y se hace necesario implementar estrategias para que esto sea más sencillo y además el tiempo en el que se obtienen resultados sea más corto, permitiendo de esta forma ahorrar recursos importantes. \\  

Para el caso presentado en el laboratorio de calidad de la energía eléctrica de la universidad industrial de Santander sede Guatigurá, es preciso implementar una plataforma que permita al investigador adquirir los datos que necesita en un menor tiempo y con mayor efectividad.  


\section{Objetivos del Proyecto}

\subsection{Objetivo general}

Establecer un sistema de adquisición de datos que permita al usuario reducir de forma significativa el tiempo y el esfuerzo empleados en las diferentes mediciones que se realizan en el laboratorio antes mencionado, además de facilitar un futuro análisis debido a la forma de presentar los datos obtenidos de una forma ordenada y concisa. 

\subsection{Objetivos específicos}

\begin{itemize}

\item Generar una plataforma de adquisición de datos compatible con la chroma 61511.   

\item Construir un switch autónomo que permita tomar mejorar la precisión de los datos adquiridos. 

\item Permitir que el usuario de la plataforma pueda manejar el sistema de forma remota con una configuración preestablecida.

\item Generar la posibilidad de que investigadores asociados puedan acceder a la información en tiempo real, es decir, puedan visualizar remotamente los resultados obtenidos en las pruebas.

\item Construir el sistema de forma modular y escalable, haciendo posible que sea mejorado y adaptado a las futuras necesidades del usuario. 

\end{itemize}








