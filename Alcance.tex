%%%% Alcance
\chapter{Alcance}
%% FALTA TERM¡NAR 
Con este trabajo se desea desarrollar una herramienta que permita el monitoreo en tiempo real de variables medidas por un prototipo de caracterización de paneles fotovoltaicos y de medición de variables metereológicas. Además del monitoreo se realizará el almacenamiento de los datos medidos, ya sea en una base de datos en la nube o en el disco duro del computador, con esta herramienta se facilitará el uso del prototipo, ya que con esto se podrá manejar remotamente. Para llevar a cabo este objetivo, se establecen que las funciones a analizar son:

\begin{description}

\item[Velocidad de transmisión de datos:] Debido a que la trasmisión de los datos se hace por medio de IOT, se debe verificar que la trasmisión de los datos sea confiable; es decir que la conexión del módulo WIFI sea estable para que el monitoreo y almacenamiento se realice adecuadamente.

\item[Funcionamiento de la aplicación móvil y plataforma \textit{online}:]Debido a que el monitoreo se hará por medio de una aplicación móvil y/o plataforma \textit{online}, se debe garantizar el correcto funcionamiento en todo momento. En la amplicación movil se realizara unicamente la visualización de los datos obtenidos por el prototipo. 

\item[Almacenamiento de datos:]Los datos enviados de las variables medidas por el prototipo, se guardan para luego ser visualizados, por esto se debe garantizar que estos queden archivados de forma adecuada y no se presenten pérdidas de información; además de que la capacidad de almacenamiento del microprocesador y plataforma online sean suficientes para cubrir los datos obtenidos.
\end{description} 

Adicional a estos, se debe tener en cuenta la alimentación de los elementos en el montaje, Ya que el prototipo se va ubicar en campo (espacio al aire libre), como recomendación, se debe tener en cuenta la alimentacion para el funcionamiento de todo el sistema. Por este motivo se debe usar un microprocesador de bajo consumo. 

 
 
 
  
  
