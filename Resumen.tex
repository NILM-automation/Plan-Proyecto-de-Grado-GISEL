%\newpage
\begin{center}
\begin{Large}
\textbf{RESUMEN DEL TRABAJO PROPUESTO}
\end{Large}
\end{center}


\begin{table}[h]
\centering
\begin{tabular}{p{3cm} p{6cm}}
%%%%%%%%%%%%%%%%%%%%%%%%%	 	TITULO		%%%%%%%%%%%%%%%%%%%%%%%% 

%	NOTA: OJO, ESCRIBIR EL TÍTULO POR PEDAZOS, NO COMPLETO UTILIZANDO "\\" PARA DAR SALTOS DE LINEA

\cline{2-2}
\multicolumn{1}{l|}{{\bf Título}}                                                                                         & \multicolumn{1}{l|}{\begin{tabular}[c]{@{}l@{}} Por establecer título. \end{tabular}}                                                                                                  \\[0.2cm] \cline{2-2} \noalign{\smallskip}
                                                                                                                         
\cline{2-2}                                                              
%%%%%%%%%%%%%%%%%%%%%%%%%	 	DIRECTOR		%%%%%%%%%%%%%%%%%%%%%%%%  
\multicolumn{1}{l|}{{\bf Director:}}                                                                                     & \multicolumn{1}{l|}{M.Sc. \insertdirector\footnotemark, \insertcorreodirector }                                                                                                           \\[0.2cm] \cline{2-2} \noalign{\smallskip}
                                                        
                                                                                                                                                               
\cline{2-2}   

                 %%%%%%%%%%%%%%%%%%%%%%%%%	 	CODIRECTOR		%%%%%%%%%%%%%%%%%%%%%%%%           
\multicolumn{1}{l|}{{\bf Codirector:}}                                                                                  & \multicolumn{1}{l|}{PhD. \insertcodirector \footnotemark, \insertcorreocodirector }                                                                                                           \\[0.2cm] \cline{2-2} \noalign{\smallskip}
                                                         \cline{2-2}                                                                  
%%%%%%%%%%%%%%%%%%%%%%%%%	 	AUTORES		%%%%%%%%%%%%%%%%%%%%%%%%                                       
\multicolumn{1}{l|}{{\bf Autores:}}                                                                                      & \multicolumn{1}{l|}{\begin{tabular}[c]{@{}l@{}}\insertautoruno\footnotemark, \insertcorreouno\\ \insertautordos \footnotemark, \insertcorreodos \end{tabular}} \\[0.4cm] \cline{2-2} \noalign{\smallskip}
                                           
\cline{2-2}                                                                                                                        
%%%%%%%%%%%%%%%%%%%%%%%%%	 	DURACIÓN		%%%%%%%%%%%%%%%%%%%%%%%%  
\multicolumn{1}{l|}{{\bf Duración:}}                                                                                         & \multicolumn{1}{l|}{6 meses. }                                                                                                                                            \\[0.2cm] \cline{2-2} \noalign{\smallskip}
                                          
\cline{2-2}                                                              
%%%%%%%%%%%%%%%%%%%%%%%%%	 	ENTIDADES		%%%%%%%%%%%%%%%%%%%%%%%%  
\multicolumn{1}{l|}{{\bf \begin{tabular}[c]{@{}l@{}}Entidades Interesadas \\en la Investigación:\end{tabular}}}          & \multicolumn{1}{l|}{\begin{tabular}[c]{@{}l@{}}
- Universidad Industrial de Santander (UIS).\\
- Escuela de Ingenierías Eléctrica, Electrónica y de \\Telecomunicaciones  
(E3T).\end{tabular}}   \\[0.4cm] \cline{2-2} \noalign{\smallskip}

\end{tabular}
\end{table}

%%%%%%%%%%%%%%%%%%%%%%%%%	 	OBJETIVOS		%%%%%%%%%%%%%%%%%%%%%%%%  
\begin{tabular}{| p{15.5cm} |}
\hline
\\
\textbf{OBJETIVO GENERAL}

Mejorar el sistema de adquisición de datos del laboratorio de calidad de la energía eléctrica UIS sede Guatiguará, a través de la implementación de un entorno de desarrollo (físico y virtual) con escalamiento, modular y con posibilidades de manejo remoto.\\

\\
\textbf{DESCRIPCIÓN DEL TRABAJO}  
\\
Este trabajo de grado se basa en la necesidad de los investigadores UIS en el área de mediciones de calidad de la energía eléctrica, debido a la falta de un sistema que les permita tomar datos de forma eficiente y precisa, esto hace que el tiempo en la adquisición de los mismos se haga tediosa y en muchas ocasiones se extienda la duración de las investigaciones en un tiempo considerable. Por lo tanto, se diseña un sistema de cierta forma automático con el cual se ingresan determinados parámetros para así obtener la mejor medición que se quiere de las diferentes variables existentes en cada una de las practicas que se realicen en el laboratorio.     

\\[0.1cm] \hline
\end{tabular}
			
%%%%%%%%%%%%%%%%%%%%%%%%%	 	PIE DE PAG		%%%%%%%%%%%%%%%%%%%%%%%%  
\footnotetext[1]{Profesor Titular E3T.}
\footnotetext[2]{Profesor Titular E3T.}
\footnotetext[3]{Estudiante de Ingeniería Electrónica de la Universidad Industrial de Santander. Código: \insertcodigouno.} 
\footnotetext[4]{Estudiante de Ingeniería Electrónica de la Universidad Industrial de Santander. Código: \insertcodigodos.} 



